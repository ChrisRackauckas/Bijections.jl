\documentclass[12pt,oneside]{amsart}
\usepackage{txfonts}
\usepackage[margin=1in]{geometry}
\usepackage{moreverb}
\nofiles

\title{Bijection type for Julia}
\begin{document}
\maketitle

\section*{Fundamentals}

This is documentation for a \verb|Bijection| datatype in Julia.  A
\verb|Bijection| behaves like an extension of a \verb|Dict| but we
ensure that the mapping from keys to values is one-to-one and we
provide an efficient way to map backwards from values to keys.

To get this module, do this (one time):
\begin{verbatim}
Pkg.clone("https://github.com/scheinerman/Bijections.jl.git")
\end{verbatim}
Then \verb|using Bijections| each session to load this module. 

To create a \verb|Bijection| we do
this:
\begin{verbatim}
julia> b = Bijection()
\end{verbatim}
in which the domain and range can be \verb|Any|. Alternatively, to
specify the types of the domain and range, use something like this:
\begin{verbatim}
julia> b = Bijection{Int, String}()
\end{verbatim}

To add pairs to \verb|b|, we can use the usual square bracket
notation:
\begin{verbatim}
julia> b[5] = "hello"
julia> b[0] = "what?"
\end{verbatim}
We cannot now define \verb|b[0]| to have another value,
nor can we define \verb|b[1]| to be an existing range element:
\begin{verbatim}
julia> b[5] = "bye"
ERROR: One of x or y already in this Bijection
 in setindex! at .....

julia> b[1] = "hello"
ERROR: One of x or y already in this Bijection
 in setindex! at .....
\end{verbatim}

To get the value associated with a given key, we use the usual square
brackets:
\begin{verbatim}
julia> print(b[0])
what?
\end{verbatim}

Because distinct keys map to distinct values, we can invert from
values to keys:
\begin{verbatim}
julia> inverse(b,"hello")
5
\end{verbatim}

If we want to change the mapping of \verb|5| to \verb|hello| we need
to first delete the pair and redefine \verb|b[5]| like this:
\begin{verbatim}
julia> delete!(b,5)
[(0,"what?")]

julia> b[5]="bye"
"bye"
\end{verbatim}



\section*{User Methods}

These are the methods that users should use. Other methods in the file
support the implementation and need not be (should not be) used.


\begin{itemize}
\item \verb|Bijection|: This is the constructor used like this:
\begin{verbatim}
b = Bijection{S,T}()
\end{verbatim}
where \verb|S| and \verb|T| are types. One can also use 
\verb|b = Bijection()| which is equivalent to
\verb|b = Bijection{Any,Any}()|. 

There is one other form: \verb|b = Bijection(x,y)| where \verb|x| and
\verb|y| are any two objects. This sets up \verb|b| in which the
domain elements have the same type as \verb|x| and whose range
elements have the same type as \verb|y|. And this initializes \verb|b|
with the pair \verb|(x,y)|. 


\item \verb|setindex!|: This is used to add a key-value pair to a
  bijection using the syntax \verb|b[x] = y|. If \verb|x| is already
  in the domain or \verb|y| is already in the range, then an error is
  raised. 

\item \verb|getindex|: This is used to query the value associated with
  a given key using the syntax \verb|b[x]|. If \verb|x| is not in the
  domain, an error is raised.

\item \verb|inverse|: This is used for the inverse mapping from values
  to keys. Syntax is \verb|inverse(b,y)| to get the key \verb|x| such
  that \verb|b[x]==y|.  If \verb|y| is not in the range, then an error
  is raised.

\item \verb|delete!|: This is used to remove a key-value pair from a
  bijection. If \verb|x| is in the domain, use \verb|delete!(b,x)|. 

\item \verb|length|: Give the number of elements in the
  bijection. Syntax is \verb|length(b)|. 

\item \verb|isempty|: Invoking \verb|isempty(b)| returns \verb|true|
  is \verb|b| has no elements; otherwise it returns \verb|false|. 

\item \verb|collect|: Returns the pairs in a bijection as an
  \verb|Array| of tuples.

\item \verb|domain|: Return, as an \verb|Array|, the elements of the
  domain of a bijection (the keys).

\item \verb|range|: Return, as an \verb|Array|, the elements of the
  range of a bijection (the values). 

\end{itemize}



\end{document}
